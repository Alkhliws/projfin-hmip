% ----------------------------------------------------------------------- %
% Pequeno texto que em poucas palavras consegue expressar o trabalho.
% O resumo deve ser concebido de forma tal que, uma pessoa ao ler o resumo
% possa entender sobre qual assunto este trabalho trata.
%
% Arquivo: resumo.tex
% ----------------------------------------------------------------------- %

\begin{resumo}
A investiga��o de tecnologias que possibilitam a mobilidade de terminais no
�mbito da Internet vem ganhando um grande espa�o na comunidade cient�fica. As
solu��es de mobilidade na camada de rede, tal como o IPv6 M�vel e seus
derivados, apresentam-se como uma forma transparente e elegante de tratar o
deslocamento de um n� m�vel entre sub-redes. Neste trabalho, descreve-se a
implementa��o de uma plataforma de testes baseada em m�quinas virtuais UML, com
fins de facilitar os estudos dos referidos protocolos, mais especificamente do
IPv6 m�vel e IPv6 m�vel hier�rquico. Visto a aus�ncia de um c�digo recente
deste �ltimo, tamb�m foi implementada uma vers�o experimental do IP M�vel
Hier�rquico. Como contribui��o adicional, s�o apresentadas an�lises preliminares
do funcionamento e desempenho dos protocolos estudados, a partir de cen�rios de
testes realizados sobre a plataforma.

\textbf{Palavras-chave:} IP M�vel, IP M�vel Hier�rquico, IP M�vel para o Linux,
Linux Modo Usu�rio, Plataforma de Testes.

\end{resumo}
