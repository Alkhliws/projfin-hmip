% ----------------------------------------------------------------------- %
% Tradu��o do resumo para a l�ngua inglesa.
% 
% 
%
% Arquivo: abstract.tex
% ----------------------------------------------------------------------- %

\begin{abstract}
The investigation of technologies that enable the mobility of terminals in
the Internet has gained a large space in the scientific community. The
solutions for mobility in the network layer, such as Mobile IPv6 and its
derivatives, appear as a transparent and elegant way to deal with the
displacement of a mobile node between subnets. This work describes the
implementation of a testing platform based on UML virtual machines with the
purpose of facilitating studies of these protocols, specifically the Mobile IPv6
and the Hierarchical Mobile IPv6. The lack of an updated code for the 
last one motivated an implementation of a experimental version of
the Hierarchical Mobile IP. As an additional contribution, it is discussed 
preliminary results on the operation and performance of the studied protocols,
extracted from test scenarios performed on the platform.

\textbf{Keywords:} Mobile IP, Mobile IP Hierarchical, Mobile IP for Linux,
User Mode Linux, Platform of tests.
\end{abstract}
