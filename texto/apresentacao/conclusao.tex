\section{Conclus�o}

\frame{
\frametitle{Trabalhos Futuros}

\begin{itemize}
\item Concep��o de um simulador da rede sem fio nas m�quinas UML, de forma
similar ao MobUML, possibilitando a an�lise dos problemas associados ao enlace e
seus impactos na camada de rede;
\item Implementa��o de modelos de mobilidade
cl�ssicos, de forma a produzir
movimentos circulares, retil�neos, randomizados, ou mesmo, a acoplagem a um
simulador de movimentos urbanos, tal como o \cite{krajzewicz};
\item Incrementos na linguagem de descri��o da UML, de forma a incorporar
aspectos de configura��o do IPv4 e facilitar a incorpora��o de novos protocolos
e \textit{deamons};
\end{itemize}

}

\frame{
\frametitle{Trabalhos Futuros}
\begin{itemize}
\item Melhorias na configura��o dos terminais virtuais e consoles, de forma a
obter Uma melhor visualiza��o do sistema por parte do usu�rio.
\item Desenvolvimento de uma interface gr�fica para a configura��o das
m�quinas
UML;
\item Cria��o autom�tica de uma conex�o virtual entre a m�quina hospedeira e
as
m�quinas virtuais; 
\item Melhorias no instalador do sistema.
\end{itemize}
}