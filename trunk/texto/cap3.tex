% ----------------------------------------------------------------------- %
% Arquivo: cap3.tex
% ----------------------------------------------------------------------- %
\chapter{Conceitos finais sobre o documento}
\label{c_cap3}

Neste cap�tulo, diferentemente do ocorreu na se��o \ref{s_c2_figuras} do cap�tulo \ref{c_cap2}, ser� apresentado uma forma para inserir tabelas no documento. A tabela \ref{t_c3_etapas} � s� um pequeno exemplo de tabela.

\begin{table}[!htpb]
\centering

\begin{small}
  \setlength{\tabcolsep}{3pt}
\begin{tabular}{|c|c|c|c|c|c|c|c|c|c|c|c|c|}\hline
 & \multicolumn{12}{c|}{Semanas}\\ \cline{2-13}
\raisebox{1.5ex}{Etapa} & 01 & 02 & 03 & 04 & 05 & 06 & 07 & 08 & 09 & 10 & 11 & 12 \\ \hline
1 & $\surd$ & $\surd$ & $\surd$ & & & & & & & & & \\ \hline
2 & & & & $\surd$ & $\surd$ & $\surd$ & $\surd$ & & & & & \\ \hline
3 & & & & & & & & $\surd$ & $\surd$ & $\surd$ & & \\ \hline
4 & & & & & & & & & & & $\surd$ & $\surd$ \\ \hline
\end{tabular} 
\end{small}
\caption{Cronograma das atividades previstas}
\label{t_c3_etapas}
\end{table} 


\section{Como usar refer�ncias bibliogr�ficas}
\label{s_c3_referencias}

O uso de cita��es ao londo do texto � uma pr�tica desej�vel. Por exemplo, em \cite{lamport94} � apresentado um documento sobre a prepara��o de textos usando \LaTeX. J� em \cite{goossens94} � apresentada uma lista de refer�ncias r�pidas para realizar as mais simples tarefas em \LaTeX.

� o caso em que voc� menciona \emph{explicitamente} o autor da refer�ncia na sentenca, algo
do tipo ``Fulano (1900)''. Neste caso o nome do autor � escrito
normalmente. Para isso use o comando \verb+\citeonline+.

A ironia ser� assim uma \ldots\ proposta  por \citeonline{lamport94}. Em \cite{exemplo} foi usado para ilustrar como uma \textit{URL} deve aparecer na se��o das refer�ncias.



