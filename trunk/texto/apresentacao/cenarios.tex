\section{Cenarios de Testes}

\frame{
\frametitle{Cen�rios Estudados}
Objetivos:
\begin{itemize}
\item Analisar os mecanismos do MIPv6 e HMIPv6;
\item Testar os seus modos de opera��o;
\item Demonstrar as potencialidades do ambiente desenvolvido;
\end{itemize}
}

\frame{
\frametitle{Metodologia para Realiza��o dos Cen�rios}
\begin{itemize}
\item Testes com \textit{ping6};
\item Captura de mensagens com \textit{tcpdump};
\item Gerador de trafego com \textit{netcat};
\end{itemize}
}

\frame{
\frametitle{An�lise dos tempos de Handover}
\begin{itemize}
\item 
\end{itemize}
}

\frame{
\frametitle{Resultados}
\begin{itemize}
\item Perfeito funcionamento do MIPv6;
\item Verificado minimamente o funcionamento do HMIPv6;
\item O tempo de handover � grande problema do MIPv6, justificando a
necessidade de extens�es como HMIPv6 e o FMIPv6;
\end{itemize}
}
